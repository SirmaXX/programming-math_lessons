
\documentclass[10pt,a4paper,twoside]{article}


\usepackage[utf8]{inputenc} %Türkçe karakterler
\usepackage{amsmath} %matematik kütüphanesi
\usepackage{graphicx} %grafik kütüphanesi


% yazar hakkında bilgiler
\title{Bilgisayar Terminolojisi}
\date{28-06-2019}
\author{Deniz Balcı}

% yazar hakkında bilgiler


% içerik hakkında bilgiler
\begin{document}
% sayfa numalandırma
  \pagenumbering{gobble}
  \maketitle
  
  
  \begin{centering}
   \section{Termonoloji} \
\end{centering} 



\begin{itemize}

  \item Algorithm(Algoritma): belli bir problemi çözmek veya belirli bir amaca ulaşmak için tasarlanan yol
  \item Application(Uygulama):Bir  Bir kuralı, ilke, bilgi veya düşünceyi yaşama geçirmek, tatbik etmek, pratiğe koymak.
  \item Artificial İntelligence(yapay zeka):Yapay zeka; insan gibi davranışlar sergileme, sayısal mantık yürütme, hareket, konuşma ve ses algılama gibi birçok yeteneğe sahip yazılımsal ve donanımsal sistemler bütünüdür.
  \item APİ(Application Programming Interface):bir uygulamaya ait yeteneklerin, başka bir uygulamada da kullanılabilmesi için, yeteneklerini paylaşan uygulamanın sağladığı arayüzdür.
  \item Bios:(Basic Input-Output System): basit (yada temel) giriş – çıkış sistemi, bir bilgisayar açılırken çalışan, değiştirilemeyen bir koddur. Bu kodun temel görevi, bilgisayara bağlı olan donanımların ve bilgisayar üzerinde çalışacak olan sistemlerin (örneğin işletim sistemi) çalışabilmesi için ortamı hazır hale getirmektir.
  \item Bug:Bir bilgisayar programı veya sistemde oluşan, istenmeyen/hatalı sonuçlara yol açan hata, kusur,
  \item Computer Science(Bilgisayar bilimleri):
  \item CPU(Central processing Unit):İşlemci veya merkezi işlemci birimi,dijital bilgisayarların veri işleyenve program komutlarını gerçekleştiren bölümdür.
  \item Code(Kod). Kod, belirli bir programlama dilinde yazılan ve makinelerle konuşmamıza yardımcı olan kurallar ve talimatlar bütünüdür
  \item Cache Memory:Sıklıkla kullanılan verileri kaydeden yapıdır.
  \item Compiler(Derleyici):Derleyici, üst seviyede yazılmış olan bir kodu, daha alt seviyelere çeviren bir programdır.
  \item Client(İstemci):Bir ağ üzerinde,sunucu bilgisayarlardan hizmet alan kullanıcı bilgisayarlardır.
  \item Debugger:Yazılan programların hataları kontrol eden yapıdır.
  \item Deep Learning(Derin öğrenme):Derin öğrenme, göreve özgü algoritmaların aksine, öğrenme verilerini temsil etmeye dayalı daha geniş bir makine öğrenme yöntemleri ailesinin bir parçasıdır.
  \item ERP(Enterprise resource planning): işletmelerin kaynaklarını (insan kaynakları fiziksel kaynaklar finansal kaynaklar) bir araya getirerek uçtan uca yönetilmesini ve verimli olarak kullanılmasını sağlamak ya da desteklemek için geliştirilmiş sistem ve yazılımların genel adıdır. 
  \item Extension(eklenti):bir yazılımın aslında yer almayan özelliklerin daha sonradan eklenen kodlarla birlikte kazandırılmasıdır(programlardan bağımsızdır)
  \item Embedded system(gömülü sistem):gömüllü sistem bilinen adıyla entegre sistem yazılım ve donanımın kombinasyonudur.
  \item GPU(Graphics processing unit):bilgisayar üzerinde görüntülenmekte olan yazı ve grafiklerin oluşturulması sırasında ekran ve işlemci arasında görev yapmakta olan dönüştürücülerdir
  \item Hardware(Donanım):İşlem yapmamızı sağlayan mekanik parçalardır (tekme atabildiğin herşey donanımdır)
  \item IDE(Integrated development environment):Entegre Uygulama Geliştirme Ortamıdır,derleyici ,yorumlayıcı,hata ayıklıcıdan oluşan yazılım geliştirme ortamıdır.

  \item Text-editor(Yazı editörü):yazı yazılmasını sağlayan program.
  \item Software(Yazılım):Bir bilgisayara özel işler  yapması için verilen komutların bütünü.
  \item SDK(Software development Kit):Yazılım geliştirme araçlarının bulunduğu uyguluma çatısı,donanım platformu,video oyun konsolu gibi şeyleri geliştirmeye izin veren bir yapıdır.
  \item Server(sunucu).Birçok kullanıcıya aynı anda hizmet sağlayan üstün özellikli  bilgisayarlardır.
  \item RAM(Random Access Memory):Rastgele erişimli bellek anlamına gelir ve  değişkenlerin, programların,açık(çalışmakta olan)dosyaların dinamik olarak depolandığı hafıza adı verilen bir donanımın bölümüdür.
  \item ROM(Read Only Memory):okunabilir bellek anlamına gelen donanımdır ancak kullanıcı veya programlar veri yazamaz
  \item Methodology(Metodoloji):(Yöntembilim), araştırmalara uygun gelen çeşitli metodları inceleyen mantık bilimidir.
  \item Terminology(Terminoloji):Terim bilimi,Bir bilim dalına ait terimlerin bütünü örn:bilgisayar termionolojisi.
  \item Syntax(söz dizimi):Temel olarak bir dilde (language) tanımlı olan öğelerin (kelime, işlem, sembol yada değerlerin) anlamlı bir dizilim oluşturmasıyla ilgilenen bilimdir.
  \item Machine learning(Makine öğrenmesi):ML, matematiksel ve istatistiksel işlemler ile veriler üzerinden çıkarımlar yaparak tahminlerde bulunan sistemlerin bilgisayarlar ile modellenmesidir.
  \item Paradigm(Paradigma):bir bilim dalı ile ilgilenen bir grup bilim insanları  tarafından ortaklaşa kabul edilen görüşlerdir.
  \item Programming Language(Programlama Dili):Bir programlama dili, insanların bilgisayara çeşitli işlemler yaptırmasına imkân veren her türlü sembol, karakter ve kurallar grubudur.
  \item Proxy:Proxy vekil sunucu anlamına gelir güvenlik amacıyla kullanılır. İnternet'e erişim sırasında kullanılan bir ara sunucudur.
  \item Network(AĞ): Bilgisayarların iletişim hatları aracılığıyla veri aktarımının sağlandığı sistem, bilgisayar ağı.
\end{itemize}

\newpage
  \begin{centering}
   \section{Kaynakça} \
\end{centering} 



\begin{itemize}
\item https://www.techopedia.com/
\item https://www.nedir.com
\item Sıfırdan başlayarak programlama öğrenme-Emre Yazıcı
\item https://www.wikipedia.org
\end{itemize}



\end{document}